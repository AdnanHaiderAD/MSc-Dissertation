\documentclass[11pt]{amsart}
\usepackage{geometry}                % See geometry.pdf to learn the layout options. There are lots.
\geometry{letterpaper}                   % ... or a4paper or a5paper or ... 
%\geometry{landscape}                % Activate for for rotated page geometry
%\usepackage[parfill]{parskip}    % Activate to begin paragraphs with an empty line rather than an indent
\usepackage{graphicx}
\usepackage{amssymb}
\usepackage{epstopdf}
\usepackage{setspace} 
\DeclareGraphicsRule{.tif}{png}{.png}{`convert #1 `dirname #1`/`basename #1 .tif`.png}

\title{Brief Article}
\author{The Author}
%\date{}                                           % Activate to display a given date or no date
\usepackage{parskip}
\setlength{\parindent}{15pt}

\begin{document}
\begin{spacing}{1.2}
\maketitle
%\section{}
%\subsection{}

\section{Motivation}

In the previous chapter, we have seen that using a MFCC representation of utterances with regions of silence removed leads to a large improvement in accuracy, time and computational complexity in the performance of DTW algorithm augmented with a  euclidean metric..The main contributing factor behind the large time and computational complexity of base line DTW is the \textbf{size} of the time series sequences. The computational cost of a DTW algorithm is $(mn)$ where $m$ and $n$ denote the length of the two time series sequences currently compared. Using longer sequences thus increases the size of the DTW cost matrix hence resulting into a greater number of computations.

The DTW algorithm on its own  is a domain independent  algorithm that  uses a similarity metric  to  implicitly extract information about global trends.  The algorithm  employes dynamic programming to search a space of mapping between the time axis of  the two respective sequences to determine the optimum alignment between them.  The only difference between MFCC-augmented DTW and baseline DTW is the feature extraction stage. In machine learning, feature extraction refers to the pre-processing stage that involves the extraction of new features from a set of raw attributes through a suitable functional mapping. The extraction phase of MFCC features  involves a segmentation of the time series followed by a functional mapping on the segmented windows. The  resultant sequence of  extracted feature vectors has a much smaller length compared to the length of the original sequence. Evident from the  experiments done in the previous chapters, it can concluded that the use mel-cepstrum features based on `cleaned' signals not only increases the accuracy of DTW but also reduces the time and computational cost through reduction of dimensionality of the original sequence.

The MFCC feature extraction is a domain dependent method.The feature extraction process can only be applied to time series sequences corresponding to speech. The purpose of this section is to investigate a self  proposed variation of DTW that 
\begin{itemize}
\item employs a domain independent functional mapping to extract relevant features 
\item employs a new-self proposed kernel to  
In this section, to tackle the problems of accuracy, time and computational complexity faced by the base line DTW, I investigate a  self-proposed data-driven methodology that can be partitioned in the following two stages:
\begin{itemize}
\item Feature extraction
\item Kernel construction
\end{itemize}  

The feature extraction stage  is motivated from the extraction of MFCCs.  The utterances are segmented  into windows of width 10 ms and functional mapping is applied to each window. The mapping  that I chose in this particular instance is as follows:






Stage 1 only increases the dimension of each point of time. The feature extraction stage does do anything to reduces the size of the time series sequence.. In the 2nd stage, I propose a kernel that computes similarity using segmented 


Limitation of value-based DTW :
\begin{itemize}
\item Ignores the context such as their positions in local features and their relations to overall trends. 
\end{itemize}

Limitation of derivative based DTW : 
\begin{itemize}
\item Fails to detect significant common sub-patterns between two sequences(mainly global trends)
\end{itemize}
Thus we need an algorithm that not gains vision over overall shapes but also on local trends.
****

Definition of local feature:
\[ f_{\mbox{local}}(r_i)= (r_i-r_{i-1}, r_i-r_{i+1})\]


Definition of global feature:
Points to consider: must reflect information about the global trends and in order to be combined with local features, they must be of the same scale.
\[ f_{\mbox{global}}(r_i)= (r_i -\sum_{k=1}^{i-1}r_k , r_i-\sum_{k=i+1}^M \frac{r_k}{M+1})\]



Kernel functions must be continuous, symmetric, and most preferably should have a positive (semi-) definite Gram matrix. Kernels which are said to satisfy the Mercer's theorem are positive semi-definite, meaning their kernel matrices have no non-negative Eigen values. The use of a positive definite kernel insures that the optimization problem will be convex and solution will be unique.


 \begin{eqnarray*}
k(x,z) &= &(x^{T}x')^2\\
&  =& (x_1z_1+x_2z_2)^2\\
&= & x_1^2z_1^2 + 2x_1z_1x_2z_2 + x_2^2z_2^2\\ 
&=& (x_1^2, �2x_1x_2, x_2^2)(z_1^2, �2z_1z_2, z_2^2)^{T}\\
&=& \phi(x)^{T}\phi(z)\\
\end{eqnarray*}

We saw that the simple polynomial kernel $k(x,z) = (x^{T}z)^2$ contains only terms of degree two. If we consider the slightly generalised kernel:
\[(x,z) = (x^{T}z+c)^2\]
with  c $> $0, then the corresponding feature mapping $\phi(x)$ contains constant and linear terms as well as terms of order two. If we generalize this notion then $k(x,x') = (x^{T}z)^M$ contains all monomials of order M. For instance, if x and z are two images, then the kernel represents a particular weighted sum of all possible products of M pixels in the first image with M pixels in the second image.
\end{spacing}
\end{document}  